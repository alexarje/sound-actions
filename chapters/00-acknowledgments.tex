\chapter*{Acknowledgments}

Many people have been of great importance to this project. Initial sparks of inspiration came from Arnt Inge Vistnes, who introduced me to the Fourier transform and laid the ground for my interest in studying music from a scientific perspective. Jøran Rudi and Bjarne Kvinnsland introduced me to the world of electroacoustic music at Notam and helped with my first music technology projects. Tellef Kvifte sparked my interest in organology, while Jon-Roar Bjørkvold and Even Ruud helped shape an understanding of the importance of embodied and cultural perspectives on musicking.

Over the years, I have enjoyed working closely with Rolf Inge Godøy at the University of Oslo. He instilled in me a curiosity about musical sound and, particularly, the thinking of Pierre Schaeffer on sound objects. This led us onto a joint adventure of exploring music-related motion, on which we have later collaborated in many different contexts. As an exchange student at the University of California, Berkeley, I was fortunate to work with computer music pioneer David Wessel. He introduced me to timbre research and artificial neural networks, but even more importantly, he showed me the joys of performing with live electronics. As a visiting Ph.D. fellow at McGill University, I was introduced to motion capture by Marcelo M. Wanderley. I also learned a lot about new interfaces for musical expression and the importance of rigorous testing and evaluation.

I am grateful for all the funding I have received throughout the years. The  Norwegian Research Council has supported several of the projects within which the ideas for this book project were developed: Musical Gestures, Sensing Music-related Actions, and Human Bodily Micromotion in Music Perception and Interaction. I should also mention the importance of the EU Cost Actions Gesture Controlled Audio Systems (ConGAS) and Sonic Interaction Design (SID), which helped build up a network of like-minded scholars around Europe. More recently, I have benefited from collaborations in the Nordic Sound and Music Computing network project funded by the Nordic Research Council.

% Bjørnar

In addition to my wonderful international colleagues, I am also fortunate to work in a music research `powerhouse' at the University of Oslo. I always learn a lot from discussions with colleagues and students at the Department of Musicology. Over the last few years, I have been fortunate to be part of building up RITMO Centre for Interdisciplinary Studies in Rhythm, Time and Motion with colleagues from Departments of Musicology, Psychology, and Informatics. Together, we aim to expand our understanding of rhythm as a fundamental property of human life. Thanks to all of you; none named, none forgotten.

%Finally, thanks to Ruth, Jorge, Israel, Sahar, Luna, Grete, Francesca, and Sebastian for
Finally, thanks to my mother, Grete, for commenting on the manuscript and giving visual guidance on the many figures in the book, to Valentina and Esperanza for showing me the joys of learning to play new instruments and bearing with me when I bring home all sorts of weird instruments for them to try out, and to Paula for continued support and encouragement.
