\chapter*{Summary}

What is a musical instrument? How do new technologies change the way we perform and perceive music? The book explores current and future approaches to musicking through the lens of musical instruments. Informed by embodied music cognition, the author proposes a model for understanding differences between traditional acoustic `sound-makers' and new electro-acoustic `music-makers.' What happens when composers build instruments, performers write code, perceivers become producers, and instruments play themselves? In this book, Alexander Refsum Jensenius presents a framework to understand how new technologies shape the future of musicking.


\section{About the Author}

Alexander Refsum Jensenius is a professor of music technology at the University of Oslo, where he co-directs RITMO Centre for Interdisciplinary Studies in Rhythm, Time and Motion. As a music researcher and research musician, he explores why music makes people move and how the human body can be used in musical interaction.
